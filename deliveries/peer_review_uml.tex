\documentclass[12pt]{article}
\usepackage[utf8]{inputenc}
\usepackage[T1]{fontenc}
\usepackage[italian]{babel}

\title{Peer-Review 1: UML}
\author{Giovanni Manfredi, Mattia Martelli, Sebastiano Meneghin\\Gruppo 27}

\begin{document}

\maketitle

Valutazione del diagramma UML delle classi del gruppo 26.

\section{Lati positivi}

Abbiamo trovato nell'utilizzo del pattern \emph{facade} una soluzione molto elegante, in quanto semplifica l'accesso ai dati contenuti nel model e nasconde
la loro effettiva implementazione a terze parti, permettendo agilmente modifiche strutturali. Abbiamo inoltre apprezzato la bassa frammentazione in classi di
concetti adiacenti, che permette dunque di avere uno schema molto compatto, pur possedendo un elevato grado di chiarezza all'atto della comprensione
dello stesso.

\section{Lati negativi}

L'utilizzo di una struttura dati dedicata per il salvataggio dell'effetto correntemente attivo sembra superfluo, poiché non è un dato, bensì un'azione
riguardante lo stato globale. Medesimo discorso vale per l'attributo \emph{winner}, in quanto la partita termina una volta individuato il vincitore,
portando ad un'eliminazione dei dati.

A giudicare dagli attributi della classe \emph{Character}, sembrerebbe esserci un fraintendimento riguardante le regole di gioco: parrebbe che il costo di
ogni carta aumenti ad ogni utilizzo, mentre dalle regole si evince che questo dovrebbe accadere solo al primo utilizzo. Per quanto riguarda la gestione
delle sottoclassi, è necessario prestare attenzione ai metodi non specificati nella classe padre, in quanto questi non saranno visibili staticamente.

Per come descritta, la classe \emph{Island} non favorisce l'unificazione delle isole, in quanto questa sembrerebbe essere delegata a terzi.  Inoltre,
l'approccio adottato per indicare la posizione della pedina \emph{mother nature}, ovvero quello di renderla un puntatore all'isola su cui si trova,
potrebbe risultare di difficile gestione ogniqualvolta si necessita un suo spostamento, in quanto si lega al discorso della gestione dell'unificazione
delle isole. Infine, l'attributo \emph{influences} sembrerebbe rappresentare informazioni parzialmente sovrapponibili con l'attibuto \emph{students},
generando potenzialmente errori legati a dati contrastanti.

La funzione \emph{addStudents} presente in \emph{Cloud}, avendo un numero fisso di parametri, potrebbe generare problemi nel caso di partite a tre
giocatori, poiché il numero di studenti è differente rispetto alle altre tipologie.

La classe \emph{Bag}, rappresentante il sacchetto in cui sono contenuti gli studenti, non permette di aumentare a runtime i contatori al suo interno
presenti: ciò rende impossibile l'implementazione di una carta personaggio.

La fusione dei concetti \emph{SchoolBoard} e \emph{Player} parrebbe essere incompleta: la \emph{DiningRoom} è difatti a sé stante, divisa in una classe
\emph{DiningTable} istanziata cinque volte, una per tavolata, dalla classe \emph{Player}. Ci sembra che ciò crei problemi di gestione facilmente
risolvibili implementando questo concetto come una mappa di tipo \(StudentColor \to int\), eliminando la necessità di avere una classe separata per ogni
contatore.

\section{Confronto tra le architetture}

Per quanto riguarda la gestione di strutture dati complesse, il loro approccio sembra concentrarsi sui tipi \emph{List} e \emph{HashMap}. In generale,
sembra collocarsi su un livello di astrazione più elevato rispetto al nostro, che si basa su \emph{Array}: ci sembrava superfluo utilizzare delle mappe
quando possiamo facilmente avere coerenza sfruttando la numerazione intrinseca dell'enumeratore \emph{StudentColor}, mentre abbiamo preferito non utilizzare
liste in quanto abbiamo trovato comodo basarci sugli indici posizionali. Ci ha fatto però riflettere sull'effettiva bontà del nostro approccio, che
potrebbe essere in certi punti aggiornato in favore di soluzioni più eleganti.

Il loro model risulta sotto certi aspetti più corposo del nostro, implementando anche funzioni che riguardano la logica del gioco. Noi abbiamo
preferito delegare tutta la logica al controller, rendendolo molto complesso, in favore di un model limitato a funzioni di rappresentazione dei dati.
Stiamo pensando di fondere i due approcci per certe funzionalità, aggiungendo al nostro model funzioni che potrebbero alleggerire il carico delle altre
componenti del progetto.

Infine, studiando il loro schema ci siamo accorti di alcuni errori da noi commessi, in particolare legati ad alcune dimenticanze riguardanti le
informazioni dei giocatori e ad un fraintendimento in relazione alle \emph{no entry tile}. Abbiamo prontamente implementato queste correzioni.

\end{document}
